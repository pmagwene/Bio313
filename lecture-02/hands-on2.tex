\documentclass{article}
\usepackage[left=1in,right=1.25in,top=1in,bottom=1in]{geometry}
\usepackage{amssymb,amsmath}
\usepackage{ifxetex,ifluatex}
\ifxetex
  \usepackage{fontspec,xltxtra,xunicode}
  \defaultfontfeatures{Mapping=tex-text,Scale=MatchLowercase}
  \setmainfont{Palatino}
  \setmonofont{Menlo}
\else
  \ifluatex
    \usepackage{fontspec}
    \defaultfontfeatures{Mapping=tex-text,Scale=MatchLowercase}
    \setmainfont{Palatino}
    \setmonofont{Menlo}
  \else
    \usepackage[utf8]{inputenc}
  \fi
\fi
% Redefine labelwidth for lists; otherwise, the enumerate package will cause
% markers to extend beyond the left margin.
\makeatletter\AtBeginDocument{%
  \renewcommand{\@listi}
    {\setlength{\labelwidth}{4em}}
}\makeatother
\usepackage{enumerate}
\usepackage{graphicx}
% We will generate all images so they have a width \maxwidth. This means
% that they will get their normal width if they fit onto the page, but
% are scaled down if they would overflow the margins.
\makeatletter
\def\maxwidth{
    \ifdim\Gin@nat@width > 3.5in 
        3.5in
    \else\Gin@nat@width
    \fi}
\makeatother
\let\Oldincludegraphics\includegraphics
\renewcommand{\includegraphics}[1]{\Oldincludegraphics[width=\maxwidth]{#1}}
\ifxetex
  \usepackage[setpagesize=false, % page size defined by xetex
              unicode=false, % unicode breaks when used with xetex
              xetex]{hyperref}
\else
  \usepackage[unicode=true]{hyperref}
\fi
\hypersetup{breaklinks=true, pdfborder={0 0 0}}
\setlength{\parindent}{0pt}
\setlength{\parskip}{6pt plus 2pt minus 1pt}
\setlength{\emergencystretch}{3em}  % prevent overfull lines

\usepackage[usenames,dvipsnames]{color}
\definecolor{lightlightgray}{gray}{0.95}
\usepackage{listings}
\lstset{
language=R,
alsolanguage=Python,
basicstyle={\ttfamily \small},                
keywordstyle=\color{BrickRed},        
stringstyle=\color{NavyBlue},       
commentstyle={\color{Apricot}\itshape},    
numberstyle=\color{LimeGreen},
backgroundcolor=\color{lightlightgray}, 
frame=none,           
framexleftmargin=1ex,                  
tabsize=4,
captionpos=b,                           
breaklines=true,                        
breakatwhitespace=false,                
showspaces=false,                       
showstringspaces=false
showtabs=false,                        
}

\setcounter{secnumdepth}{0}


\begin{document}

\section{Hands-on materials, Lecture 02}

\begin{quote}
\textbf{A note about Math fonts}: If you're viewing this material on the
course wiki, the mathematics in this document is rendered using
\href{http://www.mathjax.org/}{MathJax}. You'll need an up-to-date web
browser (IE 8+, Firefox 3.0+, Chrome) to properly render the math. If
the math looks funny in your browser, you should install the
\href{http://www.stixfonts.org/}{STIX Fonts}. The STIX fonts are
high-quality, royalty free fonts for scientific and engineering
documents. After downloading the STIX fonts, unzip the file. On OS X use
the FontBooks program to install the fonts (choose
\lstinline!File > Add Fonts! and then choose the Fonts subdirectory) and
then restart Firefox. Here's a Microsoft help page on
\href{http://www.microsoft.com/typography/TrueTypeInstall.mspx}{installing
fonts in Windows}. My experience is that of the three major browsers
Chrome does the best job of rendering MathJax math.

\end{quote}
\subsection{Vector Operations in R}

As you saw last week R vectors support basic arithmetic operations that
correspond to the same operations on geometric vectors. For example:

\begin{lstlisting}
> x <- 1:15
> y <- 10:24
> x + y             # vector addition
 [1] 11 13 15 17 19 21 23 25 27 29 31 33 35 37 39
> x - y             # vector subtraction
 [1] -9 -9 -9 -9 -9 -9 -9 -9 -9 -9 -9 -9 -9 -9 -9
> x * 3             # multiplication by a scalar
 [1]  3  6  9 12 15 18 21 24 27 30 33 36 39 42 45 
\end{lstlisting}
R also has an operator for the dot product, denoted \lstinline!%*%!.
This operator also designates matrix multiplication, which we will
discuss next week. By default this operator returns an object of the R
matrix class. If you want a scalar (or the R equivalent of a scalar,
i.e.~a vector of length 1) you need to use the \lstinline!drop()!
function.

\begin{lstlisting}
> z <- x %*% x
> class(z)      # note use of class() function
[1] "matrix"
> z
     [,1]
[1,] 1240
> drop(z)
[1] 1240
\end{lstlisting}
\begin{quote}
\textbf{Assignment 1}: In R, use the dot product operator and the
\lstinline!acos()! function to calculate the angle (in radians) between
the vectors \lstinline!x = [-3, -3, -1, -1, 0, 0, 1, 2, 2, 3]! and
\lstinline!y = [-8, -5, -3, 0, -1, 0, 5, 1, 6, 5]!.

\end{quote}
\subsection{Vector Operations in Python}

The Python equivalent of the R code above is:

\begin{lstlisting}
>>> import numpy
>>> x = numpy.arange(start=1, stop=16, step=1)
>>> y = numpy.arange(10,25) # default step = 1
>>> x
array([ 1,  2,  3,  4,  5,  6,  7,  8,  9, 10, 11, 12, 13, 14, 15])
>>> y
array([10, 11, 12, 13, 14, 15, 16, 17, 18, 19, 20, 21, 22, 23, 24])
>>> x+y
array([11, 13, 15, 17, 19, 21, 23, 25, 27, 29, 31, 33, 35, 37, 39])
>>> x-y
array([-9, -9, -9, -9, -9, -9, -9, -9, -9, -9, -9, -9, -9, -9, -9])
>>> 3*x
array([ 3,  6,  9, 12, 15, 18, 21, 24, 27, 30, 33, 36, 39, 42, 45])
>>> z = numpy.dot(x,x) # no built-in dot operator, but a dot fxn in numpy
>>> z
1240
\end{lstlisting}
Note the use of the \lstinline!numpy.arange()! function.
\lstinline!numpy.arange()! works like R's \lstinline!sequence()!
function and it returns a Numpy array. However, notice that the values
go up to but don't include the specified \lstinline!stop! value. Use
\lstinline!help()! to lookup the documentation for
\lstinline!numpy.arange()!. Python also includes a \lstinline!range()!
function that generates a regular sequence as a Python list object. The
\lstinline!range()! function has \lstinline!start!, \lstinline!stop!,
and \lstinline!step! arguments but these can only be integers. Here are
some additional examples of the use of \lstinline!arange()! and
\lstinline!range()!:

\begin{lstlisting}
>>> z = numpy.arange(1,5,0.5) 
>>> z
array([ 1. ,  1.5,  2. ,  2.5,  3. ,  3.5,  4. ,  4.5])
>>> range(1,20,2)
[1, 3, 5, 7, 9, 11, 13, 15, 17, 19]
>>> range(1,5,0.5)
Traceback (most recent call last):
  File "<stdin>", line 1, in <module>
TypeError: range() integer step argument expected, got float.
\end{lstlisting}
\subsection{Writing Functions in R}

So far we've been using R's built in functions. However the power of a
true programming language is the ability to write your own functions.

The general form of an R function is as follows:

\begin{lstlisting}
funcname <- function(arg1, arg2) {
 # one or more expressions
 # last expression is the object returned
}
\end{lstlisting}
To make this concrete, here's an example where we define a function in
the interpreter and then put it to use:

\begin{lstlisting}
> myfunc <- function(x,y){
+   x^2 + y^2     # don't type the '+' symbols, these show continuation lines
+ }

> a <- 1:5
> b <- 6:10
> a
[1] 1 2 3 4 5
> b
[1]  6  7  8  9 10
> myfunc(a,b)
[1]  37  53  73  97 125
> myfunc
function(x,y){
  x^2 + y^2
}
\end{lstlisting}
If you type a function name without parentheses R shows you the
function's definition. This works for built-in functions as well
(thought sometimes these functions are defined in C code in which case R
will tell you that the function is a `.Primitive').

\subsubsection{Putting R functions in Scripts}

When you define a function at the interactive prompt and then close the
interpreter your function definition will be lost. The simple way around
this is to define your R functions in a script that you can than access
at any time.

Choose \lstinline!File > New Script! (or \lstinline!New Document! in OS
X) in the R GUI (\lstinline!File > New > R Script! in RStudio). This
will bring up a blank editor window. Enter your function into the editor
and save the source file in your R working directory with a name like
\lstinline!vecgeom.R!.

\begin{lstlisting}
# functions defined in vecgeom.R

veclength <- function(x) {
  # Given a numeric vector, returns length of that vector
  sqrt(drop(x %*% x))
}

unitvector <- function(x) {
  # Given a numeric vector, returns a unit vector in the same direction
  x/veclength(x)
}
\end{lstlisting}
There are two functions defined above, one of which calls the other.
Both take single vector arguments. At this point there is no error
checking to insure that the argument is reasonable but R's built in
error handling will do just fine for now.

Once your functions are in a script file you can make them accesible by
using the \lstinline!source()! function (See also the
\lstinline!File > Source R code...! menu item (R GUI);
\lstinline!Edit > Source File..! in RStudio):

\begin{lstlisting}
> source("vecgeom.R")
> x <- c(-3,-3,-1,-1,0,0,1,2,2,3)
> veclength(x)
[1] 6.164414
> ux <- unitvector(x)
> ux
 [1] -0.4866643 -0.4866643 -0.1622214 -0.1622214  0.0000000  0.0000000
 [7]  0.1622214  0.3244428  0.3244428  0.4866643
> veclength(ux)
 [1] 1
\end{lstlisting}
\begin{quote}
\textbf{Assignment 2}: Write a function in R that takes two vectors,
$\vec{x}$ and $\vec{y}$, and returns a list containing the projection of
$\vec{y}$ on $\vec{x}$ and the component of $\vec{y}$ in $\vec{x}$:

\[P_{\vec{x}}(\vec{y}) = \left(\frac{\vec{x} \cdot \vec{y}}{|\vec{x}|}\right) \frac{\vec{x}}{|\vec{x}|}\]
and \[C_{\vec{x}}(\vec{y}) = \frac{\vec{x} \cdot \vec{y}}{|\vec{x}|}\]

\end{quote}
\subsection{Writing Functions in Python}

The general form of a Python function is as follows:

\begin{lstlisting}
def funcname(arg1,arg2):
    # one or more expressions
    return someresult # arbitrary python object (could even be another function)
\end{lstlisting}
An important thing to remember when writing functions is that Python is
white space sensitive. In Python code indentation indicates scoping
rather than braces. Therefore you need to maintain consistent
indendation. This may surprise those of you who have extensive
programming experience in another language. However, white space
sensitivity contributes signficantly to the readability of Python code.
Use a Python aware programmer's editor and it will become second nature
to you after a short while. I recommend you set your editor to
substitute spaces for tabs (4 spaces per tab), as this is the default
convention within the python community.

Here's an example of defining and using a function in the Python
interpreter:

\begin{lstlisting}
>>> def mypyfunc(x,y):
...     return x**2 + y**2 + 3*x*y
... 
>>> mypyfunc(10,12)
604
>>> a = numpy.arange(1,5,0.5)
>>> b = numpy.arange(2,6,0.5)
>>> mypyfunc(a,b)
array([  11.  ,   19.75,   31.  ,   44.75,   61.  ,   79.75,  101.  ,
        124.75])
>>> a = range(1,5)
>>> b = range(1,5)
>>> mypyfunc(a,b)

Traceback (most recent call last):
  File "<pyshell#52>", line 1, in -toplevel-
    mypyfunc(a,b)
  File "<pyshell#45>", line 2, in mypyfunc
    return x**2 + y**2 + 3*x*y
TypeError: unsupported operand type(s) for ** or pow(): 'list' and 'int'
>>> 
\end{lstlisting}
Note that this function works for numeric types (\lstinline!ints! and
\lstinline!floats!) as well as \lstinline!numpy.arrays! but not for
simple Python lists. If you wanted to make this function work for lists
as well you could define the function as follows:

\begin{lstlisting}
>>> def mypyfunc(x,y):
...     x = numpy.array(x)
...     y = numpy.array(y)
...     return x**2 + y**2 + 3*x*y
... 
>>> a
[1, 2, 3, 4]
>>> b
[1, 2, 3, 4]
>>> mypyfunc(a,b)
array([ 5, 20, 45, 80])
\end{lstlisting}
\subsubsection{Putting Python functions in Modules}

As with R, you can define your own Python modules that contain user
defined functions. Using a programmer's text editor, write your
function(s) and save it to a file with a \lstinline!.py! extension in a
directory in your PYTHONPATH (see {[}{[}Setting
Paths\textbar{}setting-paths{]}{]}.

\begin{lstlisting}
# functions defined in vecgeom.py
import numpy

def veclength(x):
    """Given a numeric vector, returns length of that vector""" 
    x = numpy.array(x)
    return numpy.sqrt(numpy.dot(x,x))


def unitvector(x):
    """Given a numeric vector, returns a unit vector in the same direction"""
    x = numpy.array(x)
    return x/veclength(x)
\end{lstlisting}
To access your function use an \lstinline!import! statement:

\begin{lstlisting}
>>> import vecgeom
>>> x = [-3,-3,-1,-1,0,0,1,2,2,3]
>>> help(vecgeom.veclength)
Help on function veclength in module vecgeom:

veclength(x)
    Given a numeric vector, returns length of that vector

>>> vecgeom.veclength(x)
6.164414002968976
>>> from vecgeom import * # import all functions from the vecgeom module
>>> print vecgeom.unitvector(x)
[-0.48666426 -0.48666426 -0.16222142 -0.16222142  0.          0.   
     0.16222142  0.32444284  0.32444284  0.48666426]
\end{lstlisting}
\begin{quote}
\textbf{Assignment 3}: Write Python code for the vector projection and
component functions as described in Assignment 2. In your Pweave
document illustrate the use of these functions with several examples.
Remember that your module will need to have access to the
\lstinline!numpy! module so include an appropriate \lstinline!import!
statement.

\end{quote}
\subsection{Dealing with Data Subsets in R}

Manipulating or analyzing subsets of data is one of the most common
tasks in R. The \lstinline!subset()! function comes in handy for such
operations. Consider the data set \lstinline!turtles.txt!:

\begin{lstlisting}
> turtles <- read.table('turtles.txt', header=T)
> turtles
   sex length width height
1    f     98    81     38
2    f    103    84     38
3    f    103    86     42
  # output truncated
> names(turtles)
[1] "sex"    "length" "width"  "height"
>
> # Now we'll apply the subset() function
>
> turt.sub <- subset(turtles, select = -sex)
> names(turt.sub)
[1] "length" "width"  "height"
> turt.sub
   length width height
1      98    81     38
2     103    84     38
3     103    86     42
  # output truncated
\end{lstlisting}
In the example above we create a subset of the original data set by
excluding the variable indicating the sex of each individual using the
argument \lstinline!select = -sex!. We can also explicit include only
certain variables, like this:

\begin{lstlisting}
> turt.sub2 <- subset(turtles, select=c(height,width))
> turt.sub2
   height width
1      38    81
2      38    84
3      42    86
  # output truncated    
\end{lstlisting}
\lstinline!subset()! allows you to do more than just select variables to
include. You can use the second positional argument to specify matching
criteria. For example:

\begin{lstlisting}
# gives only female turtles, all variables except sex
> female.turts <- subset(turtles, sex == "f", select = -sex)
> dim(female.turts)
[1] 24  3

# same for male turtles
> male.turts <- subset(turtles, sex == "m", select = -sex)
> dim(male.turts)
[1] 24  3

# gives only females with length > 125, all variables
> big.females <- subset(turtles, sex == "f" & length > 125)    
\end{lstlisting}
The \lstinline!subset! function is especially useful when combined with
the function \lstinline!sapply()! which allows you to apply a function
of interest to each variable. For example:

\begin{lstlisting}
> min(turtles)
Error in Summary.data.frame(..., na.rm = na.rm) : 
        only defined on a data frame with all numeric or complex variables
> min(turt.sub)  # unexpected result
[1] 35
> sapply(turt.sub, min) # here's what we were shooting for
length  width height 
    93     74     35
> sapply(female.turts, min) # for females
length  width height 
    98     81     38 
> sapply(male.turts, min) # for males
length  width height 
    93     74     35      
\end{lstlisting}
Notice how the \lstinline!min()! function chokes on the complete data
set because the function is not defined for factor variables. In the
second example \lstinline!min(turt.sub)! returns a valid result, but
also not exactly what we wanted. In this case it looked for the minimum
value across all the objects passed to it. In the third case we use the
\lstinline!sapply()! function and get the minimum on a
variable-by-variable basis. Please take a moment to look at the
documentation for the \lstinline!sapply()! function and cook up some
examples of your own.

\paragraph{Anderson's (Fisher's) iris data set}

Anderson's (or Fisher's) iris data set consists of four morphometric
measurements for specimens from three different iris species. Use the R
help to read about the iris data set (\lstinline!?iris!). We'll be using
this data set repeatedly in future weeks so familiarize yourself with
it.

\begin{quote}
\textbf{Assignment 4}: Calculate a summary table as well as correlation
and covariance matrices for each of the species in the iris data set.
Use \lstinline!help.search()! and \lstinline!apropos()! to lookup any
necessary function names.

\end{quote}
\subsection{Exploring Univariate Distributions in R}

\subsubsection{Histograms}

One of the most common ways to examine a the distribution of
observations for a single variable is to use a histogram. The
\lstinline!hist()! function creates simple histograms in R.

\begin{lstlisting}
> hist(turtles$length) # create histogram with fxn defaults
> ?hist # check out the documentation on hist
\end{lstlisting}
Note that by default the \lstinline!hist()! function plots the
frequencies in each bin. If you want the probability densities instead
set the argument \lstinline!freq=FALSE!.

\begin{lstlisting}
> hist(turtles$length,freq=F) # y-axis gives probability density
\end{lstlisting}
Here's some other ways to fine tune a histogram in R.

\begin{lstlisting}
> hist(turtles$length, breaks=12) # use 12 bins
> mybreaks = seq(85,185,8)
> hist(turtles$length, breaks=mybreaks) # specify bin boundaries   
> hist(turtles$length, breaks=mybreaks, col='red') # fill the bins with red  
\end{lstlisting}
\subsubsection{Density Plots}

One of the problems with histograms is that they can be very sensitive
to the size of the bins and the break points used. This is due to the
discretization inherent in a histogram. A `density plot' or `density
trace' is a continuous estimate of a probability distribution from a set
of observations. Because it is continuous it doesn't suffer from the
same sensitivity to bin sizes and break points. One way to think about a
density plot is as the histogram you'd get if you averaged many
individual histograms each with slightly different breakpoints.

\begin{lstlisting}
> d <- density(turtles$length)
> plot(d)    
\end{lstlisting}
A density plot isn't entirely parameter free -- the parameter you should
be most aware of is the `smoothing bandwidth'.

\begin{lstlisting}
> d <- density(turtles$length) # let R pick the bandwidth
> plot(d,ylim=c(0,0.020)) # gives ourselves some extra headroom on y-axis
> d2 <- density(turtles$length, bw=5) # specify bandwidth
> lines(d2, col='red') # use lines to draw over previous plot
\end{lstlisting}
The bandwidth determines the standard deviation of the `kernel' that is
used to calculate the density plot. There are a number of different
types of kernels you can use; a Gaussian kernel is the R default and is
the most common choice. See the documentation for more info.

The \lstinline!lattice! package is an R library that makes it easier to
create graphics that show conditional distributions. Here's how to
create a simple density plot using the \lstinline!lattice! package.

\begin{lstlisting}
> library(lattice)
> densityplot(turtles$length) # densityplot defined in lattice
\end{lstlisting}
Notice how by default the \lstinline!lattice! package also drew points
representing the observations along the x-axis. These points have been
`jittered' meaning they've been randomly shifted by a small amount so
that overlapping points don't completely hide each other. We could have
produced a similar plot, without the lattice package, as so:

\begin{lstlisting}
> d <- density(turtles$length)
> plot(d)
> nobs <- length(turtles$length)
> points(jitter(turtles$length), rep(0,nobs)) 
\end{lstlisting}
Notice that in our version we only jittered the points along the x-axis.
You can also combine a histogram and density trace, like so:

\begin{lstlisting}
> hist(turtles$length, 10, xlab='Carapace Length (mm)',freq=F)
> d <- density(turtles$length)
> lines(d, col='red', lwd=2) # red lines, with pixel width 2    
\end{lstlisting}
Notice the use of the \lstinline!freq=F! argument to scale the histogram
bars in terms of probability density.

Finally, let's some of the features of \lstinline!lattice! to produce
density plots for the `length' variable of the turtle data set,
conditional on sex of the specimen.

\begin{lstlisting}
> densityplot(~length | sex, data = turtles)    
\end{lstlisting}
There are a number of new concepts here. The first is that we used what
is called a `formula' to specify what to plot. In this case the formula
can be read as `length conditional on sex'. We'll be using formulas in
several other contexts and we discuss them at greater length below. The
\lstinline!data! argument allows us to specify a data frame or list so
that we don't always have to write arguments like
\lstinline!turtles$length! or \lstinline!turtles$sex! which can get a
bit tedious.

\subsubsection{Box Plots}

Another common tool for depicting a univariate distribution is a `box
plot' (sometimes called a box-and-whisker plot). A standard box plot
depicts five useful features of a set of observations: the median
(center most line), the upper and lower quartiles (top and bottom of the
box), and the minimum and maximum observations (ends of the whiskers).

\begin{figure}[htbp]
\centering
\includegraphics{lecture-02/boxplot-labeled.png}
\caption{A box plot represents a five number summary of a set of
observations.}
\end{figure}

There are many variants on box plots, particularly with respect to the
`whiskers'. It's always a good idea to be explicit about what a box plot
you've created depicts.

Here's how to create box plots using the standard R functions as well as
the lattice package:

\begin{lstlisting}
> boxplot(turtles$length)
> boxplot(turtles$length, col='darkred', horizontal=T) # horizontal version 
> title(main = 'Box plot: Carapace Length', ylab = 'Carapace length (mm)')
> bwplot(~length,data=turtles) # using the bwplot function from lattice
\end{lstlisting}
Note how we used the \lstinline!title()! function to change the axis
labels and add a plot title.

\paragraph{Historical note}

-- The box plot is one of many inventions of the statistician John W.
Tukey. Tukey made many contributions to the field of statistics and
computer science, particularly in the areas of graphical representations
of data and exploratory data analysis.

\subsubsection{Bean Plots}

My personal favorite way to depict univariate distributions is called a
`beanplot'. Beanplots combine features of density plots and boxplots and
provide information rich graphical summaries of single variables. The
standard features in a beanplot include the individual observations
(depicted as lines), the density trace estimated from the observations,
the mean of the observations, and in the case of multiple beanplots an
overall mean.

\begin{figure}[htbp]
\centering
\includegraphics{lecture-02/beanplot-labeled.png}
\caption{Beanplots combine features of density and box plots.}
\end{figure}

The \lstinline!beanplot! package is not installed by default. To
download it and install it use the R package installer under the
\lstinline!Packages & Data! menu (standard R GUI) or in
\lstinline!Tools > Install Packages...! in RStudio (see also the
\lstinline!Packages! tab in the lower-right window in RStudio). If this
is the first time you use the package installer you'll have to choose a
CRAN repository from which to download package info (I recommend you
pick one in the US). Once you've done so you can search for `beanplot'
from the Package Installer window. You should also check the 'install
dependencies' check box.

Once the beanplot package has been installed check out the examples to
see some of the capabilities:

\begin{lstlisting}
> library(beanplot) 
> example(beanplot)    
\end{lstlisting}
If you ran the examples in RStudio, use the \lstinline!Clear All! option
in the \lstinline!Plots! tab after running the examples in order to
reset parameters that the examples changed.

Note the use of the \lstinline!library()! function to make the functions
in the \lstinline!beanplot! library available for use. Here's some
examples of using the \lstinline!beanplot! function with the turtle data
set:

\begin{lstlisting}
> beanplot(turtles$length) # note the message about log='y'
> beanplot(turtles$length, log='') # DON'T do the automatic log transform
> beanplot(turtles$length, log='', col=c('white','blue','blue','red'))
\end{lstlisting}
In the final version we specified colors for the parts of the beanplot.
See the explanation of the \lstinline!col! argument int he beanplot
function for details.

We can also compare the carapace length variable for male and female
turtles.

\begin{lstlisting}
> beanplot(length ~ sex, data = turtles, col=list(c('red'),c('black')),
names = c('females','males'),xlab='Sex', ylab='Caparace length (mm)')
\end{lstlisting}
Note the use of the formula notation to compare the carapace length
variable for males and females. There is also a asymmetrical version of
the beanplot which can be used to more directly compare distributions
between two groups. We explore this below. Note too the use of the list
argument to \lstinline!col!, and the use of vectors within the list to
specify the colors for female and male beanplots.

We can also create a beanplot with multiple variables in the same plot
if the variables are measured on the same scale.

\begin{lstlisting}
> beanplot(turtles$length, turtles$width, turtles$height, log='',
names=c('length','width','height'), ylab='carapace dimensions (mm)') 
\end{lstlisting}
\subsection{Simple t-tests in R}

Student's t-tests can be carried out in R using the function
\lstinline!t.test()!. The \lstinline!t.test()! function can perform one
and two-sample t-tests (i.e.~comparing a sample of interest against a
hypothesized mean, or comparing the means of two samples). The
\lstinline!t.test()! function also supports a `formula' interface for
two-sample t-tests similar to the \lstinline!lm(!) function.

\begin{lstlisting}
> t.test(width ~ sex, data=turtles)

        Welch Two Sample t-test

data:  width by sex 
t = 4.7015, df = 35.355, p-value = 3.862e-05
alternative hypothesis: true difference in means is not equal to 0 
95 percent confidence interval:
  8.122699 20.460634 
sample estimates:
mean in group f mean in group m 
      102.58333        88.29167 
\end{lstlisting}
The asymmetric version of the boxplot is very useful for comparing
distributions of the same variable between two groups. To generate such
plots use the argument \lstinline!side='both'! as an argument to
\lstinline!beanplot!.

\begin{lstlisting}
> beanplot(width ~ sex, data = turtles, side='both', col=list(c('red'),c('black')))  
\end{lstlisting}
As you can see this splits the beanplot in half for each group and puts
them back to back to facilitate comparison. The difference in the mean
of the two groups is visually obvious from the beanplot.

\begin{quote}
\textbf{Assignment 5}:

\begin{itemize}
\item
  Prepare beanplots showing samples grouped by \lstinline!Species! for
  each of the quantitative variables in the iris data set. Label the x-
  and y-axes of your boxplots and give each plot a title.

  \begin{itemize}
  \item
    \textbf{Tip}: since there are three species, you can't use the
    \lstinline!side='both'! argument, and you'll need to extend the
    \lstinline!col=list! argument to add a third color.
  \end{itemize}
\item
  Carry out two-sample t-tests contrasting \lstinline!versicolor! and
  \lstinline!virginica! for each of the four morphometric variables in
  the iris data set.
  \begin{itemize}
  \item
    \textbf{Tip}: use the \lstinline!subset()! function to create a
    subset of the iris data containing just these two species.
  \end{itemize}
\item
  Write a brief paragraph interpreting the results of the t-tests you
  conducted.
\end{itemize}
\end{quote}
\subsection{Exploring Bivariate Distributions in R}

\subsubsection{Scatterplots}

When dealing with pairs of continuous variables a scatter plot is the
obvious choice. The standard \lstinline!plot! function can be used:

\begin{lstlisting}
> plot(turtles$length, turtles$width)
> plot(turtles$length ~ turtles$width)    
\end{lstlisting}
Did you notice what is different between the two versions above? You can
also use the \lstinline!data! argument with plot, like so:

\begin{lstlisting}
> plot(length ~ width, data=turtles)
\end{lstlisting}
The \lstinline!xyplot()! function from the \lstinline!lattice! package
does pretty much the same thing:

\begin{lstlisting}
> xyplot(length ~ width, data = turtles)
\end{lstlisting}
\subsubsection{Regression in R}

R has very flexible built in functions for fitting linear models.
Bivariate regression is the simplest case of a linear model.

\begin{lstlisting}
> turtles <- read.table('turtles.txt',header=T)
> names(turtles)
[1] "sex"    "length" "width"  "height"
> regr <- lm(turtles$width ~ turtles$length)
> class(regr)
[1] "lm"
> names(regr)
 [1] "coefficients"  "residuals"     "effects"       "rank"          "fitted.values"
 [6] "assign"        "qr"            "df.residual"   "xlevels"       "call"         
[11] "terms"         "model"   
> summary(regr)

Call:
lm(formula = turtles$width ~ turtles$length)

Residuals:
     Min       1Q   Median       3Q      Max 
-5.57976 -1.66578 -0.04471  1.73752  5.97104 

Coefficients:
               Estimate Std. Error t value Pr(>|t|)    
(Intercept)     19.9434     2.3877   8.353 8.99e-11 ***
turtles$length   0.6055     0.0189  32.033  < 2e-16 ***
---
Signif. codes:  0 '***' 0.001 '**' 0.01 '*' 0.05 '.' 0.1 ' ' 1 

Residual standard error: 2.654 on 46 degrees of freedom
Multiple R-Squared: 0.9571,     Adjusted R-squared: 0.9562 
F-statistic:  1026 on 1 and 46 DF,  p-value: < 2.2e-16 

> plot(turtles$width ~ turtles$length)  # scatter plot with turtles$length on x axis
> abline(regr)  # plot the regression line
\end{lstlisting}
Note the use of the function \lstinline!abline()! to plot the regression
line. Calling \lstinline!plot()! with an object of class \lstinline!lm!
shows a series of diagnostic plots. Try this.

\begin{quote}
\textbf{Assignment 6}: Write your own regression function (i.e.~your
code shouldn't refer to the built in regression functions) for mean
centered vectors in R. The function will take as it's input two vectors,
$\vec{x}$ and $\vec{y}$. The function should return:

\begin{enumerate}[1.]
\item
  a list containing the mean-centered versions of these vectors
\item
  the regression coefficient $b$ in the mean centered regression
  equation $\vec{\widehat{y}} = b\vec{x}$
\item
  the coefficient of determination, $R^2$
\end{enumerate}
Demonstrate your regression function by using it to carry out
regressions of Sepal.Length on Sepal.Width separately for the `setosa'
and `virginica' specimens from the iris data set (again,
\lstinline!subset()! is your friend). Include plots in which you use the
\lstinline!plot()! and \lstinline!abline()! functions to illustrate your
calculated regression line.

\end{quote}

\end{document}
