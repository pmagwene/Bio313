\documentclass[10pt,letterpaper]{scrartcl}
\usepackage[top=1in, bottom=1in, left=1in, right=1.25in]{geometry}

\usepackage{titlesec}
\titlespacing{\section}{0pt}{*2}{*1.5}
\titlespacing{\subsection}{0pt}{*2}{*1}
\titlespacing{\subsubsection}{0pt}{*2}{*1}

\usepackage[T1]{fontenc} 
\usepackage[utf8]{inputenc}
\usepackage[lucidasmallscale, nofontinfo]{lucimatx}
% The Lucida Fonts (http://www.pctex.com/Lucida_Fonts.html) are
% a commercially available font package for LaTeX.
% If you want to compile this document, and don't have access
% to the Lucida Fonts, comment out the line above and uncomment 
% the following lines (all the fonts below are standard in
% TeXLive 2011):
% \usepackage{paratype}
% \usepackage[scaled]{beramono}
\usepackage{microtype}
\usepackage[scaled=0.9]{DejaVuSansMono}

\usepackage{listings}
\lstMakeShortInline|

\lstset{%
aboveskip=1em,
xleftmargin=3pt,
xrightmargin=3pt,
breaklines=true,
basicstyle={\ttfamily \small},
breakatwhitespace=false,
showspaces=false,
showstringspaces=false
showtabs=false}




\usepackage{xcolor}
\definecolor{boxframe}{rgb}{0.6,0,0}
\definecolor{boxfill}{rgb}{1,.95,.95}

\lstnewenvironment{codeblock}
{\lstset{frame=single,
backgroundcolor=\color{boxfill},
rulecolor=\color{boxframe},
framesep=2.5pt,
framerule=0.5pt}}
{}

\lstnewenvironment{Code}
{}{}

\lstnewenvironment{python}
{\lstset{language=python}}
{}
\lstnewenvironment{bash}
{\lstset{language=bash}}
{}



\usepackage[colorlinks]{hyperref}

\author{Paul M.~Magwene}

\subject{Scientific Computing for Biologists}
\title{Hands-On Exercises}
\subtitle{Lecture 13: Building a Bioinformatics Pipeline, Part III}

\date{29 November 2011}

\begin{document}
\maketitle

\section*{Overview}

Last week we installed a number of bioinformatics tools and showed how they could be used to build simple analysis pipelines using bash scripts.  This week we'll build a more sophisticated pipeline using BioPython.  This pipeline will incorporate such features as web based queries and conversion of information between different file formats.

\section*{The Pipeline}


The tasks carried out by the pipeline will be as follows:

\begin{itemize}

\item Read in a nucleotide sequence from a FASTA file
\item Translate the nucleotide sequence to an amino acid sequence
\item Do a blastp search against human and fly proteins in the Swiss-Prot database using an interface to the NCBI web version of BLAST
\item Download protein sequences for the best blast hits from Swiss-Prot
\item Use MAFFT to do a multiple alignment of the original amino acid sequence and the presumed orthologs generated via the blast search
\item Analyze the query protein for known protein domains using HMMER and Pfam
%\item Use a web service to query the KEGG pathways database to look for pathways that include orthologues to your gene of interest

\end{itemize} 


You will need a working installation of Python (2.6+), IPython, and the BioPython library (1.53+) as well as the command line tools we installed last week (MAFFT, HMMER). 

\subsubsection*{The test files}

Download the file |unknown1.fas| and |unknown2.fas| from the class website. I recommend you place these in |~/tmp|.


\subsubsection*{Reading in a single sequence from a FASTA file}

As we build our pipeline I will first demonstrate the use of various modules, classes, and functions in the interactive shell and then I will give a set of functions that consolidate the commands to make them convenient to use.  

We'll start by showing how to read sequence data out of a FASTA file:
\begin{python}
>>> cd ~/tmp
>>> from Bio import SeqIO
>>> u1 = SeqIO.read('unknown1.fas','fasta')
>>> type(u1)
<class 'Bio.SeqRecord.SeqRecord'>
>>> u1
SeqRecord(seq=Seq('ATGATGAATTTTTTTACATCAAAATCGTCGAAT
CAGGATACTGGATTTAGCTCT...TGA', SingleLetterAlphabet()), 
id='YHR205W', name='YHR205W', description='YHR205W  Chr 8', dbxrefs=[])
>>> u1.name
'YHR205W'
>>> u1.description
'YHR205W  Chr 8'
>>> u1.seq
Seq('ATGATGAATTTTTTTACATCAAAATCGTCGAATCAGGATACTGG
ATTTAGCTCT...TGA', SingleLetterAlphabet())
>>> u1.seq[:10]
Seq('ATGATGAATT', SingleLetterAlphabet())
>>> u1.seq[0]
'A'
>>> u1.seq[9]
'T'
>>> u1.seq[:10].tostring()
'ATGATGAATT'
>>> u1.seq.translate()[:10]
Seq('MMNFFTSKSS', HasStopCodon(ExtendedIUPACProtein(), '*'))
\end{python}

|SeqIO| is a sub-module of the top-level module BioPython module |Bio|.  |SeqIO.read| reads a single sequence object from a file and returns an instance of a |SeqRecord| class (defined in the Biopython package). We haven't talked about classes in lecture, but a class is a programming concept that groups data and functions that operate on that data into a single object. For example, in the code above we used the |.name| and |.description| attributes to examine information about the sequence (this information was retrieved from the fasta file itself).  A |SeqRecord| holds a |Seq| object (yet another class!) as well as accessory information like the name of the sequence, a description, etc. |Seq| objects act very much like strings in terms of slicing and element access but they also have specialized function like |.translate()| that can be used to translate a nucleotide sequence into a peptide sequence. 

\subsubsection*{Reading in multiple sequences from a FASTA file}

In the code above we demonstrated how to read a single sequence from a FASTA file.  Here we demonstrate how to read multiple sequences.  The key difference is the use of the |SeqIO.parse()| function rather than |SeqIO.open|.

\begin{python}
>>> u2 = SeqIO.parse('unknown2.fas', 'fasta')
>>> type(u2)
<type 'generator'>
>>> s1 = u2.next()
>>> type(s1)
<class 'Bio.SeqRecord.SeqRecord'>
>>> s1
SeqRecord(seq=Seq('ATGTCATCAAAACCTGATACTGGTTCGGA
AATTTCTGGCCCTCAGCGACAGGAA...TGA', SingleLetterAlphabet()), 
id='YJL005W', name='YJL005W', description='YJL005W', dbxrefs=[])
>>> s1.seq
Seq('ATGTCATCAAAACCTGATACTGGTTCGGAAATTTCTGGCC
CTCAGCGACAGGAA...TGA', SingleLetterAlphabet())
>>> s2 = u2.next()
>>> s2
SeqRecord(seq=Seq('ATGTCATCAAATCATGCTATTAGTCCAGAA
ACTTCTGGCTCTCATGAGCAACAA...TGA', SingleLetterAlphabet()), 
id='MIT_Sbay_c342_13338', name='MIT_Sbay_c342_13338', 
description='MIT_Sbay_c342_13338', dbxrefs=[])
>>> s3 = u2.next()
>>> s4 = u2.next()
>>> s5 = u2.next()
---------------------------------------------------------------------------
StopIteration                             Traceback (most recent call last)
/Users/pmagwene/Desktop/tmp/<ipython console> in <module>()
StopIteration: 
\end{python}

In this case the |SeqIO.parse| function returns an object that has \emph{iterator} semantics (technically it's a `generator' but this is a technical difference that you can ignore for now). An iterator is an object that `acts like' a sequence (e.g. a list or tuple), but there are some major differences. The most important one is that an iterator does not have to compute the entire sequence at once. In the case of the |SeqIO.parse| function that means that if you have a FASTA file with thousands of sequence entries it wouldn't try to suck them all into memory. The \verb=.next()= method is used to call successive sequence entries in the FASTA file. When you call \verb=.next()= on the iterator(generator) instance you get back |SeqRecords|, one at a time. However, as the lost call demonstrates if there is no 'next' item in the iterator it raises a |StopIteration| exception. For more info about iterators and generators see Norman Matloff's  \href{http://heather.cs.ucdavis.edu/~matloff/Python/PyIterGen.pdf}{Tutorial on Python Iterators and Generators}. 

The steps for reading a FASTA sequence file can be wrapped up in the following function. We'll place each of the functions we develop in a module called \verb=pipeline.py=.  As you progress through the pipeline design you will add additional functions to this module.

\begin{python}
from Bio import SeqIO

def read_fasta(infile):
    """Read a single sequence from a FASTA file"""
    rec = SeqIO.read(infile,'fasta')
    return rec

def parse_fasta(infile):
    """Read multiple sequences from a FASTA file"""
    recs = SeqIO.parse(infile,'fasta')
    return [i for i in recs] 
\end{python}

The code above introduces another new concept called \emph{list comprehensions}. A list comprehension is a compact way of applying a function to each element in a sequence. In this case the list comprehension implicitly called |.next()| to get all the |SeqRecords| from the generator returned by |SeqIO.parse|.  You'll recall that most functions in R works in a vector-wise manner. List comprehensions provide similar semantics for Python.  Below are some simpler examples of list comprehensions. Try and predict the output of each of these before typing them in:

\begin{python}
In [1]: x = [2,4,6,8,10]
In [2]: [i**2 for i in x]
Out[2]: ???
In [3]: y = ['bob', 'tab', 'rob', 'snob']
In [4]: def juvenilize(s):
   ...:     return str(s) + "by"
   ...:
In [5]: [juvenilize(i) for i in y]
Out[5]: ???
\end{python}

You can  use the \verb=read_fasta= function as follows:
\begin{python}
>>> import pipeline
>>> recs = pipeline.parse_fasta('unknown2.fas')
>>> len(recs)
4
>>> [i.name for i in recs]
['YJL005W', 'MIT_Sbay_c342_13338', 'MIT_Smik_c333_12160', 'MIT_Spar_c300_12282']
\end{python}

Note that the \verb=parse_fasta= function will return a list of SeqRecords even when there is only a single sequence in the file. If you use the function |read_fasta| on a FASTA file with more than one sequence it will raise an error.

\subsubsection*{Translating nucleotide sequence to a protein sequence}

The next step is to translate each  DNA sequence into a corresponding protein sequence. This is very easy using the |.translate()| method associated with the |Seq| class.

\begin{python}
>>> recs[0].seq.translate()
Seq('MSSKPDTGSEISGPQRQEEQEQQIEQSSPTEANDRSIHDEV
PKVKKRHEQNSGH...ST*', HasStopCodon(ExtendedIUPACProtein(), '*'))
\end{python}

Note that the above code returns an object of type |Seq|. That's usually what we want if we're manipulating nucleotide or protein sequences but if we want to write our translated sequences back out into a file we need to create new |SeqRecords|. I illustrate this in the function below (add this to |pipeline.py|).

\begin{python}
from Bio import Seq
from Bio import SeqRecord

def translate_recs(seqrecs):
    """ nucleotide SeqRecords -> translated protein SeqRecords """ 
    proteins = []
    for rec in seqrecs:
        aaseq = rec.seq.translate()
        protrec = SeqRecord.SeqRecord(aaseq, id=rec.id, name=rec.name, 
        			      description=rec.description)
        proteins.append(protrec)
    return proteins
\end{python}    

We can then encapsulate the whole process of converting a nucleotide FASTA file to a peptide sequence FASTA file as so:

\begin{python}
def write_fasta(recs, outfile):
    ofile = open(outfile, 'w')
    SeqIO.write(recs, ofile, 'fasta')

def translate_fasta(infile, outfile):
    """ nucleotide fasta file -> protein fasta file """
    nrecs = parse_fasta(infile)
    precs = translate_recs(nrecs)
    write_fasta(precs, outfile)
\end{python}

We can then use this function from the Python interpreter like so:

\begin{python}
>>> reload(pipeline)
<module 'pipeline' from '/Users/pmagwene/synchronized/pyth/pipeline.py'>
>>> pipeline.translate_fasta('unknown2.fas', 'unknown2-protein.fasta')
\end{python}

Take a moment to open the file \texttt{unknown2-protein.fasta} in a text editor to confirm that the file now hold amino acid sequences rather than nucleotide sequences.

As an aside, what if we wanted to repeat this for a whole directory full of DNA sequences in separate FASTA files?  Here's a function to help accomplish that task:
\begin{python}
import glob

def inout_pairs(insuffix, outsuffix):
    """ Files in directory with given suffix -> list of tuples w/ (infile,outfile)"""
    infiles = glob.glob('*'+insuffix)
    pairs = []
    for infile in infiles:
        inprefix = infile[:-len(insuffix)]
        outfile = inprefix + outsuffix
        pairs.append((infile,outfile))
    return pairs                                                    
\end{python}

The |glob| module gives you filename `globbing' functionality. Globbing is a means of matching specified file or pathnames; you can think about this as a simplified class of regular expressions.  For example, you're probably familiar with command line searches like:
\begin{python}
$ ls *.fas   # list all files with the extension .fas    
$ ls unk*   # list all files that begin with 'unk'
\end{python}


The |inout_pairs| function we defined above allows us to glob file files with the given |insuffix| and create a corresponding set of names for output files. The following illustrates this:

\begin{python}
>>> pairs = pipeline.inout_pairs('.fas', '-protein.fasta')
[('unknown1.fas', 'unknown1-protein.fasta'), 
('unknown2.fas', 'unknown2-protein.fasta')]
>>> for (i,o) in pairs:
...     pipeline.translate_fasta(i,o)
...     
...  
In [27]: ls
unknown1-protein.fasta  unknown1.fas  unknown2-protein.fasta  unknown2.fas
\end{python}

Note that I changed the file suffix from `\verb=.fas=' to `\verb=.fasta=' on the output files. This isn't necessary but I find that doing so makes it easy to sort through large directories to distinguish generated files from the original files. The \verb=inout_pairs= function will come in handy when we combine our functions to generate a multi-sequence pipeline.




\subsubsection*{BLAST searches via the NCBI server}

We can use Biopython do network based BLAST searches. Here we will use blastp to search against protein sequences in the Swiss-Prot database.

\begin{python}
>>> from Bio.Blast import NCBIWWW, NCBIXML
>>> prot1 = pipeline.read_fasta('unknown1-protein.fasta')
>>> results_handle = NCBIWWW.qblast('blastp','swissprot',prot1.seq.tostring(), 
entrez_query='(Homo sapiens[ORGN]')
>>> results = results_handle.read()
>>> sfile = open('prot1_blast.out','w')
>>> sfile.write(results)
>>> sfile.close()
>>> blast_out = open('prot1_blast.out','r')
>>> brec = NCBIXML.read(blast_out)
>>> brec
<Bio.Blast.Record.Blast instance at 0x2ec22d8>
>>> len(brec.alignments) # we got 50 blast hits in the 
50
>>> brec.alignments[0]
<Bio.Blast.Record.Alignment instance at 0x2ec23a0>
>>> brec.alignments[0].accession
u'P31749'    
\end{python}

This code introduces another concept we'll call the \emph{Producer-Consumer} pattern. The Producer-Consumer pattern is a general programming concept, but the key here is that the pattern generalizes the problem of parsing complex biological data types. The producer does the work of getting the information from a file (or from the web in this case). The consumer process the information into a form we can use. In the code above the function \verb=NCBIWWW.qblast= is the producer and \verb=NCBIXML.read= plays the role of the consumer. This pattern is used over and over again in Biopython so you should spend some time trying to understand the general idea. See the Biopython tutorial for a more complete discussion.

Above we limited our query to sequences from humans. If we wanted to include all metazoan sequences we could pass \verb='(Metazoa[ORGN])'= as the argument to \verb=entrez_query=. If we didn't want to limit our search at all we would simply not include that argument (i.e. accept the default). The BLAST output is fairly complicated. See the BioPython tutorial section 7.5 for a complete breakdown of all the fields in the BLAST output.

Again, the commands above are rather involved so let's wrap them up in a function:

\begin{python}
from Bio.Blast import NCBIWWW, NCBIXML    
    
def blastp(seqrec, outfile, database='nr', entrez_query='(none)'):
    handle = NCBIWWW.qblast('blastp', database, seqrec.seq.tostring(), 
    				entrez_query=entrez_query)
    results = handle.read()
    sfile = open(outfile, 'w')
    sfile.write(results)
    sfile.close()   
    bout = open(outfile, 'r')
    brecord = NCBIXML.read(bout)    
    return brecord

def summarize_blastoutput(brecord):
    hits = []
    for alignment in brecord.alignments:
        expect = alignment.hsps[0].expect
        accession = alignment.accession
        hits.append((expect,accession))
    hits.sort() # will sort tuples by their first value (i.e. expect)
    return hits      
\end{python}

We can use this code as follows:
\begin{python}
>>> humanblast = pipeline.blastp(prot1, 'prot1-hum-blast.out', database='swissprot', 
entrez_query='(Homo sapiens[ORGN])')
>>> flyblast = pipeline.blastp(prot1, 'prot1-fly-blast.out', database='swissprot', 
entrez_query='(Drosophila melanogaster[ORGN])')
>>> humanhits = pipeline.summarize_blastoutput(humanblast)
>>> flyhits = pipeline.summarize_blastoutput(flyblast)
>>> humanhits[0]
(6.0329099999999998e-84, u'P31749')
>>> print humanhits[0][1]  # prints the swissprot accession number
P31749
>>> flyhits[0]
(3.5325700000000003e-86, u'Q8INB9')
\end{python}

Go to the Swiss-Prot \href{http://expasy.org/sprot/}{website} and use the search box to lookup those accession numbers.



\subsubsection*{Getting records from Swiss-Prot}

For a small number of accession numbers it's easy to use the web interface to Swiss-Prot. For hundred of blast hits that's just not an option. Conveniently, we can use Biopython to query the Swiss-Prot database to retrieve information about these presumed orthologs. You can access the Swiss-Prot database as follows:

\begin{python}
>>> from Bio import ExPASy
>>> from Bio import SwissProt
>>> handle1 = ExPASy.get_sprot_raw(humanhits[0][1])
>>> rec1 = SwissProt.read(handle1)
>>> print rec1.description
RecName: Full=RAC-alpha serine/threonine-protein kinase; EC=2.7.11.1; AltName: 
Full=RAC-PK-alpha; AltName: Full=Protein kinase B; Short=PKB; AltName: Full=C-
AKT;
>>> rec1.comments[0]
"FUNCTION: General protein kinase capable of phosphorylating several known 
proteins. Phosphorylates TBC1D4. Signals downstream of phosphatidylinositol 3-
kinase (PI(3)K) to mediate the effects of various growth factors such as platelet-
... output truncated ..."
>>> dir(rec1) # lets see what other attributes the record has
['__doc__', '__init__', '__module__', 'accessions', 'annotation_update', 'comments', 
'created', 'cross_references', 'data_class', 'description', 'entry_name', 'features', 
'gene_name', 'host_organism', 'keywords', 'molecule_type', 'organelle', 'organism', 
'organism_classification', 'references', 'seqinfo', 'sequence', 'sequence_length', 
'sequence_update', 'taxonomy_id']
>>> print rec1.gene_name
Name=AKT1; Synonyms=PKB, RAC;
>>> print rec1.sequence[:25] # first 25 amino acids
MSDVAIVKEGWLHKRGEYIKTWRPR
\end{python}

Here's some functions to make this more convenient:

\begin{python}
from Bio import ExPASy
from Bio import SwissProt

def get_swissrec(accession):
    handle = ExPASy.get_sprot_raw(accession)
    record = SwissProt.read(handle)
    return record
    
def swissrec2seqrec(record):
    seq = Seq.Seq(record.sequence, Seq.IUPAC.protein)
    s = SeqRecord.SeqRecord(seq, description=record.description, 
                id=record.accessions[0], name=record.entry_name)
    return s        
\end{python}

Here's an example of applying these functions:

\begin{python}
>>> reload(pipeline)
>>> ids = [humanhits[0][1], flyhits[0][1]]
>>> ids
[u'P31749', u'Q8INB9']
>>> swissrecs = [pipeline.get_swissrec(i) for i in ids]
>>> seqs = [pipeline.swissrec2seqrec(i) for i in swissrecs]
>>> seqs[0]
SeqRecord(seq=Seq('MSDVAIVKEGWLHKRGEYIKTWRPRYFLLKNDGTFIGYKERP
QDVDQREAPLNN...GTA', IUPACProtein()), id='P31749', name='AKT1_HUMAN', 
description='RecName: Full=RAC-alpha serine/threonine-protein kinase; 
EC=2.7.11.1; AltName: Full=RAC-PK-alpha; AltName: Full=Protein kinase B; 
Short=PKB; AltName: Full=C-AKT;', dbxrefs=[])
>>> seqs[1]
SeqRecord(seq=Seq('MNYLPFVLQRRSTVVASAPAPGSASRIPESPTTTGSNIINIIYSQ
STHPNSSPT...SMQ', IUPACProtein()), id='Q8INB9', name='AKT1_DROME', 
description='RecName: Full=RAC serine/threonine-protein kinase; Short=DRAC-PK; 
Short=Dakt1; Short=DAkt; EC=2.7.11.1; AltName: Full=Akt; AltName: Full=Protein 
kinase B; Short=PKB;', dbxrefs=[])
>>> seqs.append(prot1)  # add our original protein sequence to the list
>>> pipeline.write_fasta(seqs, 'unknown1-plus-human-fly.fasta')
\end{python}


\subsubsection*{Multiple sequence alignment via MAFFT}

We've now generated a new FASTA file that includes our original protein sequence and the sequences for the human and fly BLAST best hits.  We will use MAFFT to perform a multiple alignment. Biopython has built in code to simplify command line usage of common alignment programs like CLUSTALW, MAFFT, and MUSCLE.  However I'll show you how to do this with your own code using the \verb=subprocess= module.  Knowing how the |subprocess| module works is useful because it allows you to interface with any command line program from within Python.

The \verb=subprocess= module allows your Python code to start other programs (child processes) and send/get input and output from those same processes. When we use the subprocess module we're putting the Unix design element of `Everything is a file or process' to use. Here's a simple example:

\begin{python}
>>> import subprocess
>>> subprocess.call(["ls","-l"]) 
# on windows the equivalent command is
# subprocess.call(["dir",],shell=True)
total 11696
-rw-r--r--   1 pmagwene  staff    93514 Nov 22 19:36 prot1-fly-blast.out
-rw-r--r--   1 pmagwene  staff   109635 Nov 22 19:35 prot1-hum-blast.out
-rw-r--r--   1 pmagwene  staff   109635 Nov 22 19:19 prot1_blast.out
-rw-r--r--   1 pmagwene  staff     2308 Nov 22 20:07 unknown1-plus-human-fly.fasta
-rw-r--r--   1 pmagwene  staff      854 Nov 22 16:46 unknown1-protein.fasta
-rwx------   1 pmagwene  staff     2535 Nov 22 15:38 unknown1.fas
-rw-r--r--@  1 pmagwene  staff    24849 Nov 22 16:25 unknown2.fas
-rw-r--r--   1 pmagwene  staff     8331 Nov 22 16:46 unknowns-protein.fasta
\end{python}

The above code uses a convenience function \verb=call()= in the \verb=subprocess= module. We'll use the same function to run MAFFT:

\begin{python}
import subprocess

def mafft_align(infile, outfile):
    ofile = open(outfile,'w')
    retcode = subprocess.call(["mafft",infile], stdout=ofile)
    ofile.close()
    if retcode != 0:
        raise Exception("Possible error in MAFFT alignment")    
\end{python}

And we put it to use as follows:
\begin{python}
In [8]: reload(pipeline)
In [8]: pipeline.mafft_align('unknown1-plus-human-fly.fasta', 'unknown1-
alignment.fasta')
\end{python}

If all went well this should have created the file |unknown1-alignment.fasta| in your directory.  Open this alignment using JalView to examine the alignment in more detail.

\subsubsection*{Searching for protein domains using HMMER and Pfam}

As the final step of our pipeline we'll use HMMER and the Pfam database to search for known protein domains in our original protein. This assumes you have the HMMER binaries and Pfam database installed as demonstrated in last weeks exercises and that you've already run |hmmpress| against the Pfam database. Again we write a small wrapper function using the \verb=subprocess= module. This time we'll use the \verb=Popen= class to illustrate how we can capture the output produced by \verb=hmmpfam=. 

\begin{python}
def hmmer_pfam(infilename, outfilename, pfamdb):
    pipe = subprocess.Popen(["hmmscan", pfamdb, infilename], 
            stdout=subprocess.PIPE).stdout
    output = pipe.read() # this gives us the output of our command
    outfile = open(outfilename, 'w')
    outfile.write(output)
    outfile.close()
\end{python}


This function can be called like this:

\begin{python}
>>> pipeline.hmmer_pfam('unknown1-protein.fasta', 'unknown1-domains.out',
 '/Users/pmagwene/tmp/Pfam-A.hmm')
\end{python}

As before this search may take several minutes.


\subsubsection*{Putting it all together}

We've generated a variety of functions that take care of the major steps of our pipeline. It's time to put the steps together to automate the entire process. 

\begin{python}
def oneseq_pipeline(infilename, pfamdb=None,
                    compareto=['Homo sapiens','Drosophila melanogaster'],
                    skipHMMER = True,extension="XX"):
    # translate nucleotide sequence to protein seq
    protout = 'protein-' + infilename + extension     
                # add the extension so all generated files have
                # different extension than input files
                
    translate_fasta(infilename, protout)
    
    # run blastp on protein sequence against swissprot and extract best hits
    protrec = parse_fasta(protout)[0]        
    blastout ='blast-' + protout        
    besthitids = []
    for organism in compareto:
        equery = '(%s[ORGN])' % organism # create the entrez organism query
        brecord = blastp(protrec, blastout, database='swissprot', entrez_query=equery)
        bhits = summarize_blastoutput(brecord)
        besthitids.append(bhits[0][1])

    # download corresponding records from Swiss-Prot
    swissrecs = [get_swissrec(i) for i in besthitids]
    seqs = [swissrec2seqrec(i) for i in swissrecs]
    seqs.append(protrec)    
    
    # write FASTA file with best hits plus original protein sequence
    plusout = 'blasthits-' + protout
    write_fasta(seqs, plusout)     
    
    # do multiple alignment via mafft
    mafft_align(plusout, 'aligned-' + protout)

    # search for domains via HMMER/Pfam     
    if not skipHMMER:  
        if pfamdb is not None: 
            hmmerout = 'hmmer-' + protout
            hmmer_pfam(protout, hmmerout, pfamdb)

\end{python}

Our function can take as input a FASTA file with a single sequence or with multiple sequences. In the case of a multiple sequences it assumes that the `target' sequence for the search is the first sequence in the file. Also, Note the \verb=skipHMMER= argument included in the function. The HMMER search takes a relatively long time and doing it sequence by sequence is not very efficient so by default the pipeline will skip this step. If you want to include the HMMER step than specify the Pfam database file and set \verb~skipHMMER=False~.

To test out the function we do:
\begin{python}
>>> reload(pipeline)
>>> pipeline.oneseq_pipeline('unknown1.fas')
\end{python}

Let's test the pipeline using an alternate set organisms:
\begin{python}
>>> pipeline.oneseq_pipeline('unknown1.fas', 
compareto=["Homo sapiens","Mus musculus","Caenorhabditis elegans"])
\end{python}

For completeness let's also test the pipeline with the HMMER step included:
\begin{python}
>>> pipeline.oneseq_pipeline('unknown1.fas', '/home/pmagwene/tmp/Pfam-A.hmm',
skipHMMER=False)
\end{python}

Now that we're confident out single sequence pipeline function works it can be easily adapted to deal with multiple input files:

\begin{python}
def multiseq_pipeline(inext, pfamdb=None, 
                compareto=['Homo sapiens','Drosophila melanogaster'],
                skipHMMER=True):   
    inout = inout_pairs(inext, 'XX')
    infiles = [i[0] for i in inout]
    for filename in infiles:
        print "Processing %s" % filename
        oneseq_pipeline(filename, pfamdb, compareto, skipHMMER)        
\end{python}

To test the complete multi-sequence pipeline delete all the generated files (so that only \verb=unknown1.fas= and \verb=unknown2.fas= are in the unknowns directory) and try the following:
\begin{python}
>>> pipeline.multiseq_pipeline('.fas')
\end{python}

Given our example data this function will process just two input files.  However, you can add an arbitrary number of additional `.fas' files to the directory and the pipeline will process those as well with exactly the same command.

There are a number of ways the pipeline could be sped up. One obvious improvement would be to utilize a local installation of BLAST and the respective databases. However, optimization is often a complex task. The pipeline we developed here doesn't require us to install BLAST (which can be somewhat involved) and provides adequate performance for a modest number of sequences. It is possible to turn this set of Python functions into a program that you could run from the command line (rather than the Python interpeter) just like any other Unix program. 

\subsection*{python for pipeline.py}
Below is the complete code listing for the |pipeline.py| module.

\bigskip

\begin{python}
"""
An illustrative example of a bioinformatics pipeline.
Requires Python 2.6+ and BioPython 1.53+
(c) Copyright by Paul M. Magwene, 2009-2010  (mailto:paul.magwene@duke.edu)
"""
from Bio import Seq, SeqIO, SeqRecord
from Bio import ExPASy, SwissProt
from Bio.Blast import NCBIWWW, NCBIXML
    
import glob, subprocess

def read_fasta(infile):
    """Read a single sequence from a FASTA file"""
    rec = SeqIO.read(infile,'fasta')
    return rec

def parse_fasta(infile):
    """Read multiple sequences from a FASTA file"""
    recs = SeqIO.parse(infile,'fasta')
    return [i for i in recs] 
    
def write_fasta(recs, outfile):
    ofile = open(outfile, 'w')
    SeqIO.write(recs, ofile, 'fasta')    
    
def translate_recs(seqrecs):
    """ nucleotide SeqRecords -> translated protein SeqRecords """ 
    proteins = []
    for rec in seqrecs:
        aaseq = rec.seq.translate()
        protrec = SeqRecord.SeqRecord(aaseq, id=rec.id, name=rec.name, 
        			      description=rec.description)
        proteins.append(protrec)
    return proteins    
    
def translate_fasta(infile, outfile):
    """ nucleotide fasta file -> protein fasta file """
    nrecs = parse_fasta(infile)
    precs = translate_recs(nrecs)
    write_fasta(precs, outfile)    

def inout_pairs(insuffix, outsuffix):
    """ Files in directory with given suffix -> list of tuples w/ (infile,outfile)"""
    infiles = glob.glob('*'+insuffix)
    pairs = []
    for infile in infiles:
        inprefix = infile[:-len(insuffix)]
        outfile = inprefix + outsuffix
        pairs.append((infile,outfile))
    return pairs     
    
def blastp(seqrec, outfile, database='nr', entrez_query='(none)'):
    handle = NCBIWWW.qblast('blastp', database, seqrec.seq.tostring(), 
    				entrez_query=entrez_query)
    results = handle.read()
    sfile = open(outfile, 'w')
    sfile.write(results)
    sfile.close()   
    bout = open(outfile, 'r')
    brecord = NCBIXML.read(bout)    
    return brecord

def summarize_blastoutput(brecord):
    hits = []
    for alignment in brecord.alignments:
        expect = alignment.hsps[0].expect
        accession = alignment.accession
        hits.append((expect,accession))
    hits.sort() # will sort tuples by their first value (i.e. expect)
    return hits       

def get_swissrec(accession):
    handle = ExPASy.get_sprot_raw(accession)
    record = SwissProt.read(handle)
    return record
    
def swissrec2seqrec(record):
    seq = Seq.Seq(record.sequence, Seq.IUPAC.protein)
    s = SeqRecord.SeqRecord(seq, description=record.description, 
                id=record.accessions[0], name=record.entry_name)
    return s             

def mafft_align(infile, outfile):
    ofile = open(outfile,'w')
    retcode = subprocess.call(["mafft",infile], stdout=ofile)
    ofile.close()
    if retcode != 0:
        raise Exception("Possible error in MAFFT alignment")    
        
def hmmer_pfam(infilename, outfilename, pfamdb):
    pipe = subprocess.Popen(["hmmscan", pfamdb, infilename], 
            stdout=subprocess.PIPE).stdout
    output = pipe.read() # this gives us the output of our command
    outfile = open(outfilename, 'w')
    outfile.write(output)
    outfile.close()
    
def oneseq_pipeline(infilename, pfamdb=None,
                    compareto=['Homo sapiens','Drosophila melanogaster'],
                    skipHMMER = True,extension="XX"):
    # translate nucleotide sequence to protein seq
    protout = 'protein-' + infilename + extension     
                # add the extension so all generated files have
                # different extension than input files        
    translate_fasta(infilename, protout)
    
    # run blastp on protein sequence against swissprot and extract best hits
    protrec = parse_fasta(protout)[0]        
    blastout ='blast-' + protout        
    besthitids = []
    for organism in compareto:
        equery = '(%s[ORGN])' % organism # create the entrez organism query
        brecord = blastp(protrec, blastout, database='swissprot', entrez_query=equery)
        bhits = summarize_blastoutput(brecord)
        besthitids.append(bhits[0][1])

    # download corresponding records from Swiss-Prot
    swissrecs = [get_swissrec(i) for i in besthitids]
    seqs = [swissrec2seqrec(i) for i in swissrecs]
    seqs.append(protrec)    
    
    # write FASTA file with best hits plus original protein sequence
    plusout = 'blasthits-' + protout
    write_fasta(seqs, plusout)     
    
    # do multiple alignment via mafft
    mafft_align(plusout, 'aligned-' + protout)

    # search for domains via HMMER/Pfam     
    if not skipHMMER:  
        if pfamdb is not None: 
            hmmerout = 'hmmer-' + protout
            hmmer_pfam(protout, hmmerout, pfamdb)         
                        
def multiseq_pipeline(inext, pfamdb=None, 
                compareto=['Homo sapiens','Drosophila melanogaster'],
                skipHMMER=True):   
    inout = inout_pairs(inext, 'XX')
    infiles = [i[0] for i in inout]
    for filename in infiles:
        print "Processing %s" % filename
        oneseq_pipeline(filename, pfamdb, compareto, skipHMMER)                          
    
\end{python}

\end{document}
