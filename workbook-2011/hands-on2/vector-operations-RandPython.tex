%!TEX root = ./workbook-2011.tex

\section{Vector Operations in R}

As you saw last week R vectors support basic arithmetic operations that
correspond to the same operations on geometric vectors. For example:
%
\begin{R}
> x <- 1:15
> y <- 10:24
> x + y             # vector addition
 [1] 11 13 15 17 19 21 23 25 27 29 31 33 35 37 39
> x - y             # vector subtraction
 [1] -9 -9 -9 -9 -9 -9 -9 -9 -9 -9 -9 -9 -9 -9 -9
> x * 3             # multiplication by a scalar
 [1]  3  6  9 12 15 18 21 24 27 30 33 36 39 42 45 
\end{R}
%
R also has an operator for the dot product, denoted \lstinline!%*%!.
This operator also designates matrix multiplication, which we will
discuss next week. By default this operator returns an object of the R
matrix class. If you want a scalar (or the R equivalent of a scalar,
i.e.~a vector of length 1) you need to use the \lstinline!drop()!
function.

\begin{R}
> z <- x %*% x
> class(z)      # note use of class() function
[1] "matrix"
> z
     [,1]
[1,] 1240
> drop(z)
[1] 1240
\end{R}

\begin{assignment}
In R, use the dot product operator and the
\lstinline!acos()! function to calculate the angle (in radians) between
the vectors \lstinline!x = [-3, -3, -1, -1, 0, 0, 1, 2, 2, 3]! and
\lstinline!y = [-8, -5, -3, 0, -1, 0, 5, 1, 6, 5]!.
\end{assignment}

\section{Vector Operations in Python}

The Python equivalent of the R code above is:
%
\begin{python}
>>> import numpy
>>> x = numpy.arange(start=1, stop=16, step=1)
>>> y = numpy.arange(10,25) # default step = 1
>>> x
array([ 1,  2,  3,  4,  5,  6,  7,  8,  9, 10, 11, 12, 13, 14, 15])
>>> y
array([10, 11, 12, 13, 14, 15, 16, 17, 18, 19, 20, 21, 22, 23, 24])
>>> x+y
array([11, 13, 15, 17, 19, 21, 23, 25, 27, 29, 31, 33, 35, 37, 39])
>>> x-y
array([-9, -9, -9, -9, -9, -9, -9, -9, -9, -9, -9, -9, -9, -9, -9])
>>> 3*x
array([ 3,  6,  9, 12, 15, 18, 21, 24, 27, 30, 33, 36, 39, 42, 45])
>>> z = numpy.dot(x,x) # no built-in dot operator, but a dot fxn in numpy
>>> z
1240
\end{python}
%
Note the use of the \lstinline!numpy.arange()! function.
\lstinline!numpy.arange()! works like R's \lstinline!sequence()!
function and it returns a Numpy array. However, notice that the values
go up to but don't include the specified \lstinline!stop! value. Use
\lstinline!help()! to lookup the documentation for
\lstinline!numpy.arange()!. Python also includes a \lstinline!range()!
function that generates a regular sequence as a Python list object. The
\lstinline!range()! function has \lstinline!start!, \lstinline!stop!,
and \lstinline!step! arguments but these can only be integers. Here are
some additional examples of the use of \lstinline!arange()! and
\lstinline!range()!:

\begin{python}
>>> z = numpy.arange(1,5,0.5) 
>>> z
array([ 1. ,  1.5,  2. ,  2.5,  3. ,  3.5,  4. ,  4.5])
>>> range(1,20,2)
[1, 3, 5, 7, 9, 11, 13, 15, 17, 19]
>>> range(1,5,0.5)
Traceback (most recent call last):
  File "<stdin>", line 1, in <module>
TypeError: range() integer step argument expected, got float.
\end{python}