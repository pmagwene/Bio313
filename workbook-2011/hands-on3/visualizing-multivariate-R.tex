
\section{Visualizing Multivariate data in R}

Plotting and visualizing multivariate data sets can be challenge and a
variety of representations are possible. We cover some of the basic ones
here. Get the file \lstinline!yeast-subset-clean.txt! from the class
website (or use the cleaned up data set you created in the assignment
above).

\subsection{Scatter plot matrix}

A scatter plot shows the relationship between two variables by plotting
the observations in the variable space. A scatter of points that falls
approximately along a line indicate that the variables of interested are
linearly correlated, while a circular scatter indicates a lack of
correlation. Other shapes in the scatter can be indicative of non-linear
relationships. Scatter plots can also be useful for highlighting
outliers.

A scatter plot matrix is a simply a set of scatter plots, arranged like
a matrix, showing the bivariate relationships for every pair of
variables. The size of this plot is $p^2$ where $p$ is the number of
variables so you should only use it for relatively small subsets of
variables (maybe up to 7 or 8 variables at a time). The R function
\lstinline!pairs()! will create a scatter plot matrix.

\begin{R}
> yeast.clean <- read.delim("yeast-subnetwork-clean.txt")
> names(yeast.clean)
 [1] "FLO8" "RAS2" "TEC1" "PHD1" "ACE2" "SWI5" "SOK2" "RME1" "IME1" "GPA2" "MEP2" "IME2" "CLN2"
[14] "ASH1" "MUC1"
> pairs(yeast.clean[1:4]) # create a scatter plot matrix of the first 4 variables
\end{R}
\subsection{3D Scatter Plots}

A three-dimensional scatter plot can come in handy. The R library
\lstinline!lattice! has a function called \lstinline!cloud()! that
allows you to make such plots. There is also a package available on CRAN
called \lstinline!scatterplot3d! with similar functionality. I will
demonstrate in class how to install packages.

\begin{R}
> library(lattice)
> cloud(ACE2 ~ ASH1 * RAS2, data=yeast.clean)
> cloud(ACE2 ~ ASH1 * RAS2, data=yeast.clean, screen=list(x=-90, y=70)) # same plot from different angle
> attach(yeast.clean) # so we can access the variables directly
> library(scatterplot3d) # assumes package is properly installed
> scatterplot3d(ASH1, RAS2, ACE2)
> scatterplot3d(ASH1, RAS2, ACE2, angle=-30)
\end{R}
See the help file for \lstinline!cloud()! and \lstinline!panel.cloud()!
for information on setting parameters.

\subsection{Colored grid plots}

A colored grid (or `heatmap') is another way of representing 3D data. It
most often is used to represent a variable of interest as a function of
two parameters. Grid plots are created using the \lstinline!image()!
function in R.

\begin{R}
> x <- seq(0, 2*pi, pi/20)
> y <- seq(0, 2*pi, pi/20)
> coolfxn <- function(x,y){
+    cos(x) * cos(y)}
> z <- outer(x,y,coolfxn) # the outer product of two matrices or vectors, see docs
> dim(z)
[1] 41 41
> image(x,y,z)
\end{R}
The \lstinline!x! and \lstinline!y! arguments to \lstinline!image()! are
vectors, the \lstinline!z! argument is a matrix (in this case created
using the outer product operator in conjunction with our function of
interest).
