%!TEX root = ./hands-on1.tex

\subsection{Sweave for R}

Sweave documents weave together documentation/discussion and code into a
single document. The pieces of code and documentation are referred to as
`chunks'. R comes with a set of tools that allow you to extract just the
code, or to turn the entire document into a nicely formatted report.

Here's a simple Sweave document to get you started:
%
\begin{noindentcodeblock}[tex]
\documentclass{article}
\begin{document}

This is a very simple Sweave file. It includes only a single code chunk.

<<>>=
z <- rnorm(30, mean=0, sd=1)
summary(z)
@

That code chunk generated a random sample of 30 observations drawn from a normal distribution with mean zero and standard deviation one.

\end{document}  
\end{noindentcodeblock}
%
Type those lines into a text editor and save the document with the name
|sweave1.Rnw|.

Let's break down the various pieces of the document. The first two lines
and the last line represent LaTeX~commands.

\begin{Code}{tex}
\documentclass{article}
\begin{document}
    ....
\end{document}
\end{Code}
For simple Sweave documents that's all the LaTeX you need to learn.
However, learning a little bit more about the document preparation
system gives you the option of producing very nicely formatted output as
we'll see in a little bit.

The R code is preceeded by the text |<<>>=|. This tells Sweave
that you're starting a code chunk. The |@| symbol after the
code chunk tells Sweave that you're going back to writing documentation
chunks.

If you already have a working installation of LaTeX on your computer you
can now compile this into a nicely formatted document from the R
interpretter. Change the R working directory so that it's in the same
directory where you saved the |sweave1.Rnw| file. The execute
the following commands:
%
\begin{R}
> library(tools)  # makes the texi2dvi function available
> Sweave('sweave1.Rnw') # compiles our Sweave document into 'sweave1.tex'
> texi2dvi("sweave1.tex", pdf = TRUE)
\end{R}
%
These commands will produce two new documents -- |sweave1.tex|
and a PDF document, |sweave1.pdf|. You can open the
|sweave1.tex| file in any text editor and you'll see that it's
just a slightly modified version of the |sweave1.Rnw| file you
created. The PDF file contains the nicely formatted report.

If you got an error message the most likely reason is that R can't find
the path to your LaTeX executable. To check this try:
%
\begin{R}
> texi2dvi('sweave1.tex', pdf=TRUE, quiet=FALSE)
\end{R}
%
If that's the case, you can fix that by typing the following into the R
command line (on OS X):
%
\begin{R}
> Sys.setenv("PATH" = paste(Sys.getenv("PATH"),"/usr/texbin",sep=":"))  
\end{R}
%
Then try executing the |texi2dvi| command again. If that
solved your problem you can make this permanent by adding that line to
your |.Rprofile| (located in |/Users/yourname| on \OSX; if the
file doesn't already exist go ahead and create it). If that doesn't work
please see me for troubleshooting help.

\subsubsection{RStudio makes Sweaving easy!}

RStudio hides some of the complexity of Sweaving documents. Simply
create or open your Sweave document (use the .Rnw extension) in RStudio
and then hit the |Compile PDF| button. If your LaTeX setup is
working, and the document and code are valid, Rstudio will compile
everything behind the scenes and pop up a nice PDF.

\subsubsection{A fancier Sweave document}

Let's get a little bit fancier and show how we can create graphics and
use some LaTeX formatting features to produce a nicer document.

\begin{Code}{latex}
\documentclass[letterpaper]{article}
\usepackage[margin=0.75in]{geometry}

\title{My Second Sweave Report}
\author{John Q. Public}

\begin{document}
\maketitle
This is a still a simple Sweave file. However, now it includes several code chunks and several \LaTeX\ specific commands.

\section{Sampling from the random normal distribution}

<<>>=
z <- rnorm(30, mean=0, sd=1)
summary(z)
@

That code chunk generated a random sample of 30 observations drawn from a normal distribution with mean zero ($\mu = 0$) and standard deviation one ($\sigma = 1$).

\section{Generating figures}

We can also automatically imbed graphics in our report. For example, the following will generate a histogram.

% this tells Sweave to set the graphics 
% to be half the width of the text
\setkeys{Gin}{width=0.5\textwidth} 
<<fig=TRUE>>=
hist(z)
@

\end{document}    
\end{Code}
%
Notice how we put an argument, |fig=TRUE| within the second
code chunk delimiter. This will tell Sweave to automatically imbed a
figure with the histogram graphic we created into our report. Save this
as |sweave2.Rnw| and repeat the above steps to compile it into
a PDF report.

For a full overview of Sweave's capabilities see the documentation for
Sweave availabe at
\url{http://www.stat.uni-muenchen.de/~leisch/Sweave/}.

\subsection{Pweave for literate programming in Python}

Pweave uses almost exactly the same syntax as Sweave to delimit code and
document chunks. Here's a simple Pweave document.

\begin{Code}{latex}
\documentclass[letterpaper]{article}
\usepackage[margin=0.75in]{geometry}
\usepackage{graphicx} % unlike Sweave, Pweave doesn't
                      % pull this in automatically

\title{My First Pweave Report}
\author{John Q. Public}

\begin{document}
\maketitle
This is a still a simple Pweave file. As in our Sweave example,
there are several code chunks and we've included a figure.

\section{Sampling from the random normal distribution}

<<>>=
from numpy import random
z = random.normal(loc=0, scale=1, size=30)
@

That code chunk generated a random sample of 30 
observations drawn from a normal distribution with mean 
zero ($\mu = 0$) and standard deviation one ($\sigma = 1$).

\section{Generating figures}

We can also automatically imbed graphics in our 
report. For example, the following will generate 
a histogram.

% this tells Sweave to set the graphics 
% to be half the width of the text
\setkeys{Gin}{width=0.5\textwidth} 
<<fig=True>>=
import pylab
pylab.hist(z)
@

\end{document}
\end{Code}
%
Note that the second code chunk, we wrote |fig=True| in the
Pweave document, whereas we wrote |fig=TRUE| for the Sweave
document. This minor difference reflects the different syntax for
boolean values in R and Python.

Save that code in a text file called |pweave1.Pnw| and from
the bash shell (\emph{not} in the Python interpretter) type the
following command:
%
\begin{bash}
Pweave -f "tex" pweave1.Pnw
\end{bash}
%
The option |-f "tex"| tells Pweave to output a file (Note:
from the Windows command prompt you must use double quotes around
``tex'', on Unix-based systems either single our double quotes work
fine). Assuming you got no error messages, you can then compile this to
PDF using the following command:
%
\begin{bash}
pdflatex pweave1.tex
\end{bash}
