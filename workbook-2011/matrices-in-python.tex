
\subsection{Matrices in Python}

Matrices in Python are created are created using the
\lstinline!Numeric.array()! function. In Python you need to be a little
more aware of the type of the arrays that you create. If the argument
you pass to the \lstinline!array()! function is composed only of
integers than Numeric will assume you want an integer matrix which has
consequences in terms of operations like those illustrated below. To
make sure you're matrix has floating type values you can use the
argument \lstinline!typecode=Numeric.Float!.

\begin{python}
>>> import numpy as np # I'm 'aliasing' the name so I can type 'np' instead of 'numpy'
>>> array = np.array # setup another alias
>>> X = array(range(1,13))
>>> X
array([ 1,  2,  3,  4,  5,  6,  7,  8,  9, 10, 11, 12])
>>> X.shape = (4,3) # rows, columns
>>> X
array([[ 1,  2,  3],
       [ 4,  5,  6],
       [ 7,  8,  9],
       [10, 11, 12]])
>>> 1/X # probably not what you expected
array([[1, 0, 0],
       [0, 0, 0],
       [0, 0, 0],
       [0, 0, 0]])
>>> X = array(range(1,13), dtype=np.float)
>>> X.shape = 4,3
>>> X
array([[  1.,   2.,   3.],
       [  4.,   5.,   6.],
       [  7.,   8.,   9.],
       [ 10.,  11.,  12.]])
>>> 1/X # that's more like it
array([[ 1.        ,  0.5       ,  0.33333333],
       [ 0.25      ,  0.2       ,  0.16666667],
       [ 0.14285714,  0.125     ,  0.11111111],
       [ 0.1       ,  0.09090909,  0.08333333]])
>>> X
array([[  1.,   2.,   3.],
       [  4.,   5.,   6.],
       [  7.,   8.,   9.],
       [ 10.,  11.,  12.]])
>>> X + X
array([[  2.,   4.,   6.],
       [  8.,  10.,  12.],
       [ 14.,  16.,  18.],
       [ 20.,  22.,  24.]])
>>> X - X
array([[ 0.,  0.,  0.],
       [ 0.,  0.,  0.],
       [ 0.,  0.,  0.],
       [ 0.,  0.,  0.]])
>>> np.dot(X,np.transpose(X)) # dot fxn in numpy gives matrix multiplication for arrays
array([[  14.,   32.,   50.,   68.],
       [  32.,   77.,  122.,  167.],
       [  50.,  122.,  194.,  266.],
       [  68.,  167.,  266.,  365.]])
>>> np.identity(4)
array([[1, 0, 0, 0],
       [0, 1, 0, 0],
       [0, 0, 1, 0],
       [0, 0, 0, 1]])
>>> np.sqrt(X)
array([[ 1.        ,  1.41421356,  1.73205081],
       [ 2.        ,  2.23606798,  2.44948974],
       [ 2.64575131,  2.82842712,  3.        ],
       [ 3.16227766,  3.31662479,  3.46410162]])
>>> np.cos(X)
array([[ 0.54030231, -0.41614684, -0.9899925 ],
       [-0.65364362,  0.28366219,  0.96017029],
       [ 0.75390225, -0.14550003, -0.91113026],
       [-0.83907153,  0.0044257 ,  0.84385396]])
\end{python}

The code above also demonstrated the Numpy functions \lstinline!dot()!,
\lstinline!transpose()! and \lstinline!identity()!. Note too that Numpy
has a variety of functions such as \lstinline!sqrt()!and
\lstinline!cos()! that work on an element-wise basis.

Indexing of arrays in Numpy is demonstrated below. You'll see that
Python arrays support `slicing' operations. For more on slicing and
other array basics see the Numpy documentation at
\href{http://docs.scipy.org/doc/}{http://docs.scipy.org/doc/}.

\begin{python}
>>> X
array([[  1.,   2.,   3.],
       [  4.,   5.,   6.],
       [  7.,   8.,   9.],
       [ 10.,  11.,  12.]])
>>> X[0,0] # get the 0th row, 0th column (remember that Python sequences are zero-indexed!)
1.0
>>> X[3,0] # get the fourth row, 1st column
10.0
>>> X[:2,:2]  # an example of slicing, get the first two columns and rows (i.e. indices 0 and 1)
array([[ 1.,  2.],
       [ 4.,  5.]])
>>> X[1:,:2] # get everything after the 0th row and  the first two columns
array([[  4.,   5.],
       [  7.,   8.],
       [ 10.,  11.]])
\end{python}
To calculate matrix inverses in Python you need to import the
\lstinline!numpy.linalg! package.

\begin{python}
>>> import numpy.linalg as la
>>> import numpy.random as ra  # for matrices with elements from random distributions
>>> C = ra.normal(loc=0,scale=1,size=(4,4)) # do help(ra.normal) for explanation of argumnets
>>> C
array([[ 0.79525679,  1.11730719, -2.19257712, -0.06289276],
       [ 0.7087366 ,  0.70574975, -1.51599336, -0.90360945],
       [-0.33845153, -0.20109722, -0.75245988, -0.56027025],
       [-0.51692665,  0.59972543,  1.55562234,  1.88639367]])
>>> Cinv = la.inv(C)
>>> np.dot(C, Cinv) # again result is approx the identity matrix due to floating point precision
array([[ 1.00000000e+000, -5.55111512e-017, -6.93889390e-017,  2.94902991e-017],
       [ 1.11022302e-016,  1.00000000e+000, -1.11022302e-016, -5.55111512e-017],
       [ 1.11022302e-016, -2.22044605e-016,  1.00000000e+000,  2.77555756e-017],
       [ 0.00000000e+000, -4.44089210e-016,  0.00000000e+000,  1.00000000e+000]])
>>> print np.array2string(np.dot(C,Cinv),precision=2, suppress_small=True)
[[ 1. -0.  0.  0.]
 [-0.  1.  0.  0.]
 [ 0.  0.  1.  0.]
 [-0. -0. -0.  1.]]
\end{python}