
\section{Descriptive statistics as matrix functions}

Assume you have a data set represented as a $n \times p$ matrix $X$ with
observations in rows and variables in columns. Below I give formulae for
calculating some descriptive statistics as matrix functions.

\subsection{Mean vector and matrix}

To calculate a row vector of means, $\mathbf{m}$: 
\[
\mathbf{m} = \frac{1}{n} \mathbf{1}^T  X
\] where $1$ is a $n \times 1$ vector of ones.

A $n \times p$ matrix $M$ where each column is filled with the mean
value for that column is: 
\[
M = \mathbf{1}\mathbf{m}
\]

\subsection{Deviation matrix}

To re-express each value as the deviation from the variable means
(i.e.~each columns is a mean centered vector) we calculate a deviation
matrix: 
\[
D = X - M
\]

\subsection{Covariance matrix}

The $p \times p$ covariance matrix is given by: 
\[
S = \frac{1}{n-1} D^T D
\]

\subsection{Correlation matrix}

The correlation matrix, $R$, can be calculated from the covariance
matrix by: 
\[
R = V S V
\]

where $V$ is a $p \times p$ diagonal matrix where
$V_{ii} = 1/\sqrt{S_{ii}}$.

\subsection{Concentration matrix and Partial Correlations}

If the covariance matrix, $S$ is invertible, than inverse of the
covariance matrix, $S^{-1}$, is called the `concentration matrix' or
`precision matrix'. We can relate the concentration matrix to partial
correlations as follow. Let 
\[
P = S^{-1}
\]
Then:
\[
\mbox{cor}(x_i,x_j \mid X \setminus \{x_i,x_j\}) = -\frac{p_{ij}}{\sqrt{p_{ii} p_{jj}}}
\]

where $X \backslash \{x_i,x_j\}$ indicates all variables other than
$x_j$ and $x_i$. You can read this as `the correlation between x and y
conditional on all other variables.'

\medskip
\begin{assignment}
\small
The data set |yeast-subnetwork-raw.txt| (see class website), consists of gene expression measurements
for 15 genes from 173 two-color microarray experiments (see Gasch et al.
2000). These genes are members of a gene regulatory network that determines how yeast cells
respond to nitrogen starvation. The values in the data set are
expression ratios (treatment:control) that have been transformed by
applying the $\log_2$ function (so that a ratio of 1:1 has the value 0,
a ratio of 2:1 has the value 1, and a ratio of 1:2 has the value 0.5).    
    
\par\medskip    
The raw data file
\lstinline!yeast-subnetwork-raw.txt! has the genes (variables) arranged
by rows and the observations (experiments) in columns. There are also
missing values. Using R, show how to read in the data set and then
create a matrix where the genes are in columns and the observations in
rows. Then replace any missing values (\lstinline!NA!) in each column
with the variable (gene) means (there are better ways to impute missing
values but this will do for now). Encapsulate these steps in a \emph{generic} function that will work with any data file with the same organization as that above.

\par\medskip
Functions that might come in handy for this assignment 
include: \lstinline!read.delim()!, \lstinline!t()!, \lstinline!subset()!,
\lstinline!as.matrix()!, and \lstinline!is.na()!. Note that
\lstinline!t()! applies to data frames as well as matrices. Also take
note of the \lstinline!na.rm! argument of \lstinline!mean()!. You might consider creating a function that handles the missing value
replacement and using it in conjunction with the \lstinline!apply()!
function. \lstinline!colnames()! and \lstinline!rownames()! allow you to
assign/extract column and row names for a matrix. Use the
\lstinline!write.table()! function to save your results (I recommend you
use \lstinline!"\t"! (i.e.~tab) as the \lstinline!sep! argument).
\end{assignment}


\medskip
\begin{assignment}
Create an R library that includes functions that
use matrix operations to calculate each of the descriptive statistics
discussed above (except the concentration matrix / partial
correlations). Calculate these statistics for the
\lstinline!yeast-subnetwork! data set and check the results of your
functions against the built-in R functions.
\end{assignment}

